\documentclass[a4paper]{article}
\usepackage[left=2cm,right=2cm,top=1cm,bottom=1cm]{geometry}
\usepackage{graphicx}
\usepackage{fontawesome}
\usepackage{titlesec}
\usepackage{xeCJK}
\usepackage{tikz}
\usepackage[OT1]{fontenc} 
\usepackage{mathptmx}
\usepackage{enumitem}
%\usepackage{layout}
%\usepackage{xcolor}

%\definecolor{mygray}{gray}{.9}

\newcommand{\mysec}[1]{%
\begin{tikzpicture}[baseline=(a.base)]
 \node[rounded corners=1ex,fill=gray!20] (a) {#1};
\end{tikzpicture}
}


\setmainfont{Times New Roman}
\setCJKmainfont[BoldFont={SimHei}]{SimSun}
\pagestyle{empty}

\titleformat{\section}{\Large}{}{0em}{\mysec}[\vspace{0.5ex}\titlerule] 

\setlength\parindent{0em}

\begin{document}
%\layout

\newlength{\mycode}
\newlength{\mycodet}
\settowidth{\mycode}{Mail:bxg620@icloud.com}
\settowidth{\mycodet}{individual CV}

\begin{minipage}[b]{\mycodet}
\makebox[\mycodet][l]{\Huge \bf 白旭光}\\[2ex]
\makebox[\mycodet][l]{\large 个人简历}
\end{minipage}
\hfill
\begin{minipage}[b]{\mycode}
%\makebox[\mycode][c]{\large 求职意向:研发管理培训生}
\makebox[\mycode][c]{\large 求职意向:\phantom{工艺工程师}}
\end{minipage}
\hfill
\begin{minipage}[b]{\mycode}
\makebox[\mycode][r]{\faPhone\,18200509859}\\
\makebox[\mycode][r]{\faEnvelopeO\,bxg620@live.com}
\end{minipage}
\hfil
\begin{minipage}[b]{2.5cm}
\includegraphics{potraits.JPG}
\end{minipage}

\vspace{0.5cm}

\section{个人基本信息}
\hspace{-9pt}
\begin{tabular}{lll}
出生日期:1994. 9 & 籍\qquad 贯:山西忻州 & 家庭所在地:山西忻州 \\
政治面貌:共青团员 & 最高学历:硕士研究生 & 研究方向:真空电子技术
\end{tabular}

\section{教育背景}
2011. 9 - 2015. 7\hspace{1cm}桂林电子科技大学\hspace{2em}光电信息工程学士\\
2015. 9 - 2017. 10\hspace{0.8cm}电子科技大学\hspace{4em}光学工程硕士\hspace{2em}2018年6月底毕业

\section{课程信息}
本科:模拟电路,数字电路,微机原理,面向对象编程,光学教程,通信原理,固体物理\\
研究生:光电成像导论,液晶光电子学,显示技术导论,光电信息检测,半导体光电子学

\section{个人技能}
CET-4,CET-6\\
计算机二级,掌握51单片机的应用,熟练掌握C/C++语言\\
%熟悉ZEMAX软件的使用\\
可通过ANSYS软件进行热力学分析\\
可通过Opera3D软件进行基本电子光学系统的设计和仿真分析 

\section{项目经历}
\begin{itemize}[leftmargin=*]
\item \textbf{项目名称}:空心阴极的设计与制作\\
\textbf{起止时间}:2016.9-2017.10\\
\textbf{项目背景}:用电推进来取代传统的化学推进使得小型在轨卫星的轨道调整和保持更加方便和经济\\
\textbf{项目应用}:为电推进组件提供稳定的电子源,以促进放电室内的电离\\
\textbf{知识背景}:阴极电子学,气体放电,等离子体,电动力学和热力学\\
\textbf{个人负责内容}:器件的仿真分析。用ANSYS分析器件内热分布,计算不同结构下加热器的加热效率;用Opera3D软件仿真器件内的电场分布和电子的运动与分布。利用朗缪尔探针测试实验中羽流区的电子密度,分析放电的实际情况。
\item \textbf{项目名称}:单晶LaB$_6$-FEA微观形貌调控与电子发射特性研究\\
\textbf{起止时间}:2017.1-2017.12\\
\textbf{项目应用}:制作用于微焦点电子枪的场发射阵列阴极\\
\textbf{知识背景}:阴极电子学,光刻,电化学\\
\textbf{个人负责内容}:硅掩蔽层的微米级光刻工艺。通过在单晶LaB$_6$基底的硅掩蔽层上进行光刻工艺来达到印制阵列形貌的目的。
\end{itemize}

\section{获奖情况}
2015年10月获研究生三等学业奖学金

\section{个人评价}
兴趣广泛。喜欢了解不同国家的历史和文化;喜欢计算机编程,了解并会使用Linux系统,掌握C/C++、Python及Lua等编程语言的基础应用;喜欢电子游戏。性格较内向,热爱生活,乐于助人。

%\section{个人评价}
%性格较内向,热爱生活,乐于助人,做事认真


\end{document}
